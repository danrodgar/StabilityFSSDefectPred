\chapter{Conclusions and Future Work}
\label{cha:concl-y-line}

\section{Conclusions}
\label{sec:conclusions}

The concept of \acrlong{fs} was a term unknown to me. It seemed very interesting since there are algorithms that help reduce computational resources focused on data mining within the scope of machine learning. This has been a project in which the internal function of these algorithms has not been deepened, but knowing different examples of them.

Personally the knowledge acquired will help me both on a personal and the professional level. The relationship that this topic has with Big Data, a very developed concept today in the professional world, is great since both concepts are based on the use of large quantities of data.

The objectives established in Section \ref{sec:objectives} have been met. First, the basic concepts that help create a basis for understanding the experimental work part have been known. Different defect prediction datasets have been sought and six of them have been chosen to perform the tests with \acrlong{fs} algorithms. In R, packages have been searched to make \acrlong{fs} and have been used with the dataset to know the differences in the outputs of these algorithms. The rest of the objectives have also been met: how to measure stability and complexity, and know the concept of SHAP values.

Finally, conclude with the difference both of the definition and in the result returned from the different proven algorithms. The amount of combinations between the algorithms that can be made according to the desired result has also captured my attention.

\section{Future Work}
\label{sec:future-work}

This project does not come to an end here. You can continue to learn much more about the selection of attributes and metrics that help to know the importance of its outputs.

Specifically the works that can be continued in the future after having carried out this project are:

\begin{itemize}
    \item Study the rest of the \acrshort{fs} packages that are in R. The ones known in this project are only five chosen among all those that are in the \acrshort{cran}.
    
    \item spFSR package uses mlr package to create the task, the learner and the measure. This package is in 'maintenance-only' mode since July 2019. But there is a package, mlr3 that is under development and replaces mlr. Future work would be to adapt the use of spFSR to the mlr3 package.
    
    \item A package can be created in R to collect the acquired knowledge in such a way that it is stored openly for all users.
\end{itemize}
