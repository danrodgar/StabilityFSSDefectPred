\chapter*{Resumen}
\label{cha:resumen}
\markboth{Resumen}{Resumen}

\addcontentsline{toc}{chapter}{Resumen}

La selección de atributos consiste en elegir entre las variables de entrada aquellas que son más relevantes. Existen diferentes algoritmos que se encargan de seleccionar las características que les parecen relevantes. No todos eligen las mismas características como importantes. Diferentes algoritmos pueden seleccionar diferentes variables. Dependiendo del algoritmo utilizado, las características seleccionadas cambian. Puede medir qué tan estables son esos algoritmos de selección de atributos en un conjunto de datos determinado. En este proyecto se conocerán diferentes paquetes en R para realizar la selección de atributos, se probarán con diferentes conjuntos de datos y se aplicarán diferentes métricas a los resultados.

\textbf{Palabras clave:} \myThesisKeywords.